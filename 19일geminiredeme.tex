\documentclass{article}
\usepackage[utf8]{inputenc}
\usepackage{kotex}
\usepackage{geometry}
\geometry{a4paper, margin=1in}

\begin{document}

\section*{LeDeuxions.com 사이트 콘텐츠 보강 작업 요약 (2025-12-19)}

본 문서는 Gemini에 의해 수행된 5개 HTML 파일의 주요 변경 사항을 요약합니다.
모든 변경은 Google AdSense 승인 기준 충족 및 사용자 경험 향상을 목표로 합니다.

\subsection*{1. 공통 변경 사항 (5개 파일 모두 적용)}
\begin{itemize}
    \item \textbf{상단 네비게이션 바 추가:} 사용자가 사이트의 주요 섹션(Home, Web Projects, Contact)을 쉽게 탐색할 수 있도록 \texttt{<nav>} 태그를 이용한 공통 헤더를 추가했습니다. (AdSense의 '쉬운 탐색' 요건 충족)
    \item \textbf{기본 CSS 스타일 적용:} 가독성 향상 및 깔끔한 페이지 레이아웃을 위해 기본적인 CSS 스타일을 \texttt{<style>} 태그 내에 추가했습니다. 모바일 환경에서도 적절히 표시되도록 반응형 \texttt{viewport} 메타 태그를 포함했습니다.
    \item \textbf{공통 Footer 적용:} 모든 페이지 하단에 저작권 정보 및 Privacy/Terms/Contact 링크가 포함된 일관된 Footer를 적용했습니다.
    \item \textbf{HTML 구조 표준화:} 모든 파일에 \texttt{lang="ko"} 속성을 명시하고, 시맨틱한 HTML 구조를 적용하여 웹 표준을 준수하도록 했습니다.
\end{itemize}

\subsection*{2. 파일별 주요 변경 사항}
\begin{itemize}
    \item \textbf{index.html (메인 페이지):}
    \begin{itemize}
        \item 사이트의 정체성을 설명하는 'About LeDeuxions.com' 섹션을 추가했습니다.
        \item 'LeDeuxions'로서의 실험적 작업과 '현실의 필요'에 의한 실용적 작업을 분리하여 설명하는 '두 갈래의 작업' 문단을 추가하여 사이트의 복합적인 성격을 명확히 했습니다.
    \end{itemize}
    \item \textbf{web-projects/index.html:}
    \begin{itemize}
        \item 페이지 상단에 이 곳이 어떤 프로젝트들을 모아두는 허브 페이지인지 설명하는 안내 문단을 추가했습니다.
        \item 하위 프로젝트로 연결되는 링크 예시를 \texttt{<ul>} 목록 형태로 추가하여 페이지의 역할을 명확히 했습니다.
    \end{itemize}
    \item \textbf{privacy/index.html:}
    \begin{itemize}
        \item 'Privacy Policy'라는 명확한 제목을 추가했습니다.
        \item 정보 수집 최소화 원칙, Google AdSense 광고 및 쿠키 사용, 제3자 정보 제공 없음 원칙에 대한 구체적인 내용을 문단별로 추가하여 정책의 명확성을 높였습니다.
    \end{itemize}
    \item \textbf{terms/index.html:}
    \begin{itemize}
        \item 'Terms of Service'라는 명확한 제목을 추가했습니다.
        \item 서비스가 '있는 그대로(As-Is)' 제공된다는 책임의 한계 조항과, 악의적 사용을 금지하는 사용자 의무 조항을 추가하여 운영 원칙을 명시했습니다.
    \end{itemize}
    \item \textbf{contact/index.html:}
    \begin{itemize}
        \item 'Contact'라는 명확한 제목을 추가했습니다.
        \item 개인 운영 사이트로서 모든 문의에 대한 즉각적인 답변이 어려울 수 있음을 안내하는 문단을 추가하여 사용자의 기대를 관리하고 신뢰도를 높였습니다.
        \item 문의할 이메일 주소를 명확히 표시했습니다.
    \end{itemize}
\end{itemize}

\end{document}
